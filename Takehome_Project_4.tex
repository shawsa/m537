\documentclass[12pt]{article}

%%%%%%%%%%%%%%%%%%%%%%%%%%%%%%%%%%%%%%%%%%%%%%%%%%%%%%%%%%%%%%%%%%%%%%%%%%%%%%%%%%%%%%%%%%%%%%%%%%%%
% Math
\usepackage{fancyhdr} 
\usepackage{amsfonts}
\usepackage{amsmath}
\usepackage{amssymb}
\usepackage{amsthm}
%\usepackage{dsfont}

%%%%%%%%%%%%%%%%%%%%%%%%%%%%%%%%%%%%%%%%%%%%%%%%%%%%%%%%%%%%%%%%%%%%%%%%%%%%%%%%%%%%%%%%%%%%%%%%%%%%
% Macros
\usepackage{calc}

%%%%%%%%%%%%%%%%%%%%%%%%%%%%%%%%%%%%%%%%%%%%%%%%%%%%%%%%%%%%%%%%%%%%%%%%%%%%%%%%%%%%%%%%%%%%%%%%%%%%
% Commands and Custom Variables	
\newcommand{\problem}[1]{\hspace{-4 ex} \large \textbf{Problem #1} }
\let\oldemptyset\emptyset
\let\emptyset\varnothing
\newcommand{\norm}[1]{\left\lVert#1\right\rVert}
\newcommand{\sint}{\text{s}\kern-5pt\int}
\newcommand{\powerset}{\mathcal{P}}
\renewenvironment{proof}{\hspace{-4 ex} \emph{Proof}:}{\qed}
\newcommand{\RR}{\mathbb{R}}
\newcommand{\NN}{\mathbb{N}}
\newcommand{\QQ}{\mathbb{Q}}
\newcommand{\ZZ}{\mathbb{Z}}
\newcommand{\CC}{\mathbb{C}}
\renewcommand{\Re}{\operatorname{Re}}
\renewcommand{\Im}{\operatorname{Im}}

\renewcommand{\vec}[1]{\boldsymbol{#1}}


%%%%%%%%%%%%%%%%%%%%%%%%%%%%%%%%%%%%%%%%%%%%%%%%%%%%%%%%%%%%%%%%%%%%%%%%%%%%%%%%%%%%%%%%%%%%%%%%%%%%
%page
\usepackage[margin=1in]{geometry}
\usepackage{setspace}
%\doublespacing
\allowdisplaybreaks
\pagestyle{fancy}
\fancyhf{}
\rhead{Shaw \space \thepage}
\setlength\parindent{0pt}

%%%%%%%%%%%%%%%%%%%%%%%%%%%%%%%%%%%%%%%%%%%%%%%%%%%%%%%%%%%%%%%%%%%%%%%%%%%%%%%%%%%%%%%%%%%%%%%%%%%%
%Code
\usepackage{listings}
\usepackage{courier}
\lstset{
	language=Python,
	showstringspaces=false,
	formfeed=newpage,
	tabsize=4,
	commentstyle=\itshape,
	basicstyle=\ttfamily,
}

%%%%%%%%%%%%%%%%%%%%%%%%%%%%%%%%%%%%%%%%%%%%%%%%%%%%%%%%%%%%%%%%%%%%%%%%%%%%%%%%%%%%%%%%%%%%%%%%%%%%
%Images
\usepackage{graphicx}
\graphicspath{ {images/} }
\usepackage{float}

%tikz
\usepackage[utf8]{inputenc}
\usepackage{pgfplots}
\usepgfplotslibrary{groupplots}

%%%%%%%%%%%%%%%%%%%%%%%%%%%%%%%%%%%%%%%%%%%%%%%%%%%%%%%%%%%%%%%%%%%%%%%%%%%%%%%%%%%%%%%%%%%%%%%%%%%%
%Hyperlinks
%\usepackage{hyperref}
%\hypersetup{
%	colorlinks=true,
%	linkcolor=blue,
%	filecolor=magenta,      
%	urlcolor=cyan,
%}

\begin{document}
	\thispagestyle{empty}
	
	\begin{flushright}
		Sage Shaw \\
		m537 - Spring 2018 \\
		\today
	\end{flushright}
	
{\large \textbf{Take-home Project 4}}\bigbreak

%%%%%%%%%%%%%%%%%%%%%%%%%%%%%%%%%%%%%%%%%%%%%%%%%%%%%%%%%%%%%%%%%%%%%%%%%%%%%%%%%%%%%%%%%%%%%%%%%%%%
\problem{(2) $\iff $ (4)} \\
\begin{proof}
	Obviously (4) $\implies$ (2). It remains to show that (2) $\implies$ (4). \bigbreak
	
	Suppose that $L$ is continuous at $\vec{0}$. Let $\{\vec{x}_n\} \subseteq X$ such that $\lim\limits_{n \to \infty} \vec{x}_n = \vec{x} \in X$. Then clearly $\lim\limits_{n \to \infty} (\vec{x}_n - \vec{x}) = \vec{0}$.
	\begin{align*}
	\vec{0} & = L\vec{0} \\
	& = \lim_{n \to \infty} L(\vec{x}_n - \vec{x}) \\
	& = \lim_{n \to \infty} (L\vec{x}_n - L\vec{x}) \\
	& = (\lim_{n \to \infty} L\vec{x}_n) - L\vec{x}
	\end{align*}
	Finally $\lim\limits_{n \to \infty} L\vec{x}_n = L\vec{x}$.	
\end{proof}

%%%%%%%%%%%%%%%%%%%%%%%%%%%%%%%%%%%%%%%%%%%%%%%%%%%%%%%%%%%%%%%%%%%%%%%%%%%%%%%%%%%%%%%%%%%%%%%%%%%%
\problem{(2) $\implies$ (3)} \\
\begin{proof}
	Suppose $L$ is continuous at $\vec{0}$. From class we have the alternative definition of continuity that says for $\epsilon=1$ there exists a $\delta >0$ such that if $\norm{\vec{x}}<\delta$ then $\norm{L\vec{x}}<1$. \bigbreak
	
	Let $\vec{x} \in X$. Then $\frac{\delta}{2\norm{\vec{x}}}\vec{x} \in X$ and $\norm{\frac{\delta}{2\norm{\vec{x}}}\vec{x}} = \frac{\delta}{2\norm{\vec{x}}}\norm{\vec{x}} = \frac{\delta}{2} < \delta$. Thus $\norm{L\frac{\delta}{2\norm{\vec{x}}}\vec{x}} < 1$. Finally this implies that $\norm{L\vec{x}} < \frac{2}{\delta}$.
\end{proof}

%%%%%%%%%%%%%%%%%%%%%%%%%%%%%%%%%%%%%%%%%%%%%%%%%%%%%%%%%%%%%%%%%%%%%%%%%%%%%%%%%%%%%%%%%%%%%%%%%%%%
\problem{(3) $\implies$ (4)} \\
\begin{proof}
	Suppose there exists $A>0$ such that for all $\vec{x} \in X$, $\norm{L\vec{x}} \leq A \norm{\vec{x}}$. \bigbreak
	
	We would like to show $L$ is continuous at all points in $X$. \bigbreak
	
	Let $x \in X$ and $\{\vec{x}_n\}_{n=1}^\infty \subseteq X$ such that $\lim\limits_{n\to \infty} \vec{x}_n = \vec{x}$. Let $\epsilon >0$ be given. Since $\lim\limits_{n\to \infty} \vec{x}_n = \vec{x}$ there exists $N \in \NN$ such that if $n\geq N$ then $\norm{\vec{x}_n - \vec{x}} < \frac{\epsilon}{A}$. Then for any $n \geq N$
	\begin{align*}
		\norm{L \vec{x_n} - L\vec{x}} & = \norm{L(\vec{x}_n - \vec{x})} \\
		& \leq A \norm{\vec{x}_n - \vec{x}} \\
		& < A \frac{\epsilon}{A} = \epsilon
	\end{align*}
\end{proof}



%%%%%%%%%%%%%%%%%%%%%%%%%%%%%%%%%%%%%%%%%%%%%%%%%%%%%%%%%%%%%%%%%%%%%%%%%%%%%%%%%%%%%%%%%%%%%%%%%%%%
\problem{(4) $\implies$ (1)} \\
\begin{proof}
	Let $L$ be continuous at all points in $X$. Let $\{x_n\}_{n=1}^\infty \subseteq X$ such that $\lim_{n \to \infty} \vec{x}_n = \vec{0}$. We would like to show that there exists $C>0$ such that $\norm{L\vec{x}_n} \leq C$ for all $n \in \NN$. \bigbreak
	
	Since $L$ is continuous at all points in $X$ we know that $L$ is continuous at $\vec{0}$. Thus there exists $N \in \NN$ such that if $n \geq N$ then $\norm{L\vec{x}_n} < 1$. Then the set $\{ L\norm{\vec{x}_n} \}_{n=1}^N$ is a finite set of real numbers. Let $C_1$ be the maximum of this set. Let $C = max\{C_1, 1\}$. \bigbreak
	
	This $C$ satisfies our condition. Consider a particular $\vec{x}_n$. If $n<N$ then $\norm{L\vec{x}_n} \leq C_1 \leq C$, otherwise $\norm{L\vec{x}_n} \leq 1 \leq C$.
\end{proof}

%%%%%%%%%%%%%%%%%%%%%%%%%%%%%%%%%%%%%%%%%%%%%%%%%%%%%%%%%%%%%%%%%%%%%%%%%%%%%%%%%%%%%%%%%%%%%%%%%%%%
\problem{(1) $\implies$ (2)} \\
\begin{proof}
	Suppose (1). We would like to show that $L$ is continuous at $\vec{0}$. \bigbreak
	
	
\end{proof}

%%%%%%%%%%%%%%%%%%%%%%%%%%%%%%%%%%%%%%%%%%%%%%%%%%%%%%%%%%%%%%%%%%%%%%%%%%%%%%%%%%%%%%%%%%%%%%%%%%%%



\end{document}
